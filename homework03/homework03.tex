\documentclass{homework}
\usepackage{cancel}
\usepackage{amsthm}
\usepackage{cleveref}
\usepackage{upgreek}
\usepackage[framed]{mcode}
\usepackage{mathrsfs}
\usepackage{tikz}
\usepackage{units}
\usetikzlibrary{matrix}
\newtheorem{lemma}{Lemma}

\title{Kevin Joyce}
\course{Math 512 - Integral Equations Homework 3}
\author{Kevin Joyce}
\docdate{\today}
\begin{document} 
\newcommand{\figref}[1]{\figurename~\ref{#1}}
\renewcommand{\bar}{\overline}
\renewcommand{\hat}{\widehat}
\renewcommand{\SS}{\mathcal S}
\renewcommand{\NN}{\mathcal N}
\newcommand{\DD}{\mathcal D}
\newcommand{\eps}{\varepsilon}
\newcommand{\TTheta}{\overline{\underline \Theta} }
\newcommand{\del}{\partial}
\newcommand{\approxsim}{\overset{\cdotp}{\underset{\cdotp}{\sim}}}

\problem{Show that any contraction operator $A$ is:
\begin{enumerate}[(a)]
  \item Continuous, i.e. for any convergent sequence $y_n\to y$ the sequence $A y_n \to Ay$;
  \item ``Bounded'' in the following sense: for any bounded sequence $y_n$ the sequence $Ay_n$ is bounded.
\end{enumerate}}

\begin{solution}[(a)]
  Let $0\le c < 1$  such that $\|Ay_n - Ay\| \le c\|y_n - y\|$ given by $A$ a contraction.  Hence $\|Ay_n - Ay\| \to 0$ at least as fast as $\|y_n - y\|\to 0$.
\end{solution}

\begin{solution}[(b)]
  Let $\|y_n\| \le B$ for all $n$ and $c$ as above.  Then 
  $$
    \|Ay_n\| = \|Ay_n -A0 + A0\| \le \|Ay_n - A0\| + \|A0\| \le cB + \|A0\|.
  $$
\end{solution}

\problem{ Construct the Neumann series for the Volterra equation of the second kind
$$
  y(x) = \lambda \int_0^x y(s) ds + 1
$$
and find the solution.}

\begin{solution}
  Let $A$ be the integral operator given by $Ay = \int_0^x y(s)ds$. The Neumann series is given by 
  $$
    y_n = \sum_{k=0}^{n-1} \frac{(x\lambda)^k}{k!}.
  $$
  To see this, the inductive step is given by
  $$
    y_{n+1} = \lambda \int_0^x y_n ds + 1 = \lambda \int_0^x \sum_{k=0}^{n-1} \frac{(x\lambda)^k}{k!}ds + 1 = \lambda \sum_{k=0}^{n-1} \frac{\lambda^k s^{k+1}}{(k+1)!} + 1 = \sum_{k=0}^n \frac{(x\lambda)^k}{k!}.
  $$
  This is precisely the series for $e^{\lambda x}$.
\end{solution}

\problem{ Construct the resolvent kernel for the equation in Problem 2 and use it to find the solution. }

\begin{solution}
  The $n$th term for resolvent kernel is given by
  $$
    K_n(x,s) = \frac{(x-s)^{n-1}}{(n-1)!}.
  $$
  To see this, the inductive step is 
  $$
    K_{n+1}(x,s) = \int_s^x K_n(x,t) 1\,dt = \int_s^x \frac{(x-t)^{n-1}}{(n-1)!}dt = \left.-\frac{(x-t)^{n}}{n!}\right|_{t=s}^x = \frac{(x-s)^{n}}{n!}.
  $$
  Note that $\sum \lambda^{n-1} K_n$ converges absolutely and uniformly to $e^{\lambda(x-s)}$, hence the solution is given by
  $$
    y = 1 + \lambda R_\lambda\,1 = 1 + \lambda \int_0^x e^{\lambda(x-s)}\,1\,ds = 1 + \lambda e^{\lambda x} \left(\left.\frac {-1}\lambda e^{-\lambda s}\right|_{s=0}^x\right) = e^{\lambda x}.
  $$
\end{solution}

\problem{ Reduce the equation in Problem 2 to a Cauchy problem and find the solution. }
\begin{solution}
  By differentiating both sides of the equation and noting $y(0) = 1$, we see that the problem is equivalent to the ordinary differential equation
  $$
    y' = \lambda y,\quad y(0) = 1.
  $$
  By inspection, this simple equation has the solution $y(x) = e^{\lambda x}$.
\end{solution}
\end{document} 

