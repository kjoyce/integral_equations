\documentclass{homework}
\usepackage{cancel}
\usepackage{amsthm}
\usepackage{cleveref}
\usepackage{upgreek}
\usepackage[framed]{mcode}
\usepackage{mathrsfs}
\usepackage{tikz}
\usepackage{units}
\usetikzlibrary{matrix}
\newtheorem{lemma}{Lemma}

\title{Kevin Joyce}
\course{Math 512 - Integral Equations Homework 4}
\author{Kevin Joyce}
\docdate{\today}
\begin{document} 
\newcommand{\figref}[1]{\figurename~\ref{#1}}
\renewcommand{\bar}{\overline}
\renewcommand{\hat}{\widehat}
\renewcommand{\SS}{\mathcal S}
\renewcommand{\NN}{\mathcal N}
\newcommand{\DD}{\mathcal D}
\newcommand{\eps}{\varepsilon}
\newcommand{\del}{\partial}

\problem{Construct the Neumann series for the Fredholm equation of the second kind
$$
  y(x) = \lambda \int_0^1y(s)\,ds + f(x)
$$
for ``small'' $\lambda$ and $f(x)= 1$ and find the solution.
}

\begin{solution}
  Let $A$ be the integral operator $\int_0^1 \cdot\, ds$. For small enough $\lambda$, the map $y\to \lambda Ay + f$ is a contraction whose fixed point is the solution and is given by repeated composition to an arbitrary function $y_0$.  Successive applications of this map to $y_0 = 0$ yields the Neumann series
  \begin{equation}
    y_n = f + \lambda Af + \lambda^2 A^2 f + \dots + \lambda^n A^n f. \label{neumann}
  \end{equation}
  Note that
  \begin{equation}
    A^n f = \int_0^1 \dots \int_0^1 1 ds = 1. \label{kern1}
  \end{equation}
  So
  $$
    y = \lim_{n\to\infty} y_n = \sum_{k=0}^\infty \lambda^k = \frac{1}{1-\lambda}
  $$
  for $|\lambda| < 1$.

  We can easily verify that $y$ is a solution,
  $$
    y = \frac{1}{1-\lambda} = \frac{\lambda + 1 - \lambda}{1-\lambda} = \lambda \frac{1}{1-\lambda} + 1 = \lambda A y + f.
  $$
  \end{solution}

\problem{ Construct the resolvent kernel for the above equation using various approaches: }

  \subproblem{Assume that $\lambda$ is ``small''. For what $\lambda$ does the series for the resolvent kernel converge? }

\begin{solution}
  We take $f \in C[0,1]$. Observe that $ \|Ay\| \le \|y\|(1-0) =
  \|y\|$ and $\|A 1\| = 1$, hence $\|A\| = 1$.  Therefore, absolute
  and uniform convergence in \eqref{neumann} is guaranteed when
  $\|A\||\lambda|=|\lambda|<1$. Taking the limit in
  \eqref{neumann}, using the result that $A^n$ is an integral operator with a given kernel $k_n$, and interchanging $\int$ with $\sum$, the
  resolvent operator is given by
  \begin{equation}
  y = f + \lambda \int_0^1 \sum_{n=0}^\infty \lambda^n k_{n+1}(x,s) f(s) ds \label{neumann_res}
  \end{equation}
  Using similar reasoning as in \eqref{kern1}, terms in the series of the resolvent kernel are given by 
$$
    \lambda^n k_{n+1}(x,s) = \lambda^n \int_0^1 k_n(x,s)k_n(s,x) = \lambda^n \cdot 1,
  $$
  Continuing from \eqref{neumann_res}, the resolvent is
  \begin{equation}
  y = f +  \frac{\lambda}{1-\lambda} \int_0^1 f(s) ds. \label{neumann_resolv}
  \end{equation}
\end{solution}

  \subproblem{Use the fact that the kernel is symmetric and express the resolvent through the characteristic values and eigenfunctions.}

  \begin{solution}
    The characteristic values and eigenfunctions of $A$ satisfy
    $$
      \lambda \int_0^1 \phi_k(s)\, ds = \phi_k(x).
    $$
    Hence all the eigenfunctions for $A$ are constant functions, and the one orthonormal
    eigenfunction is $\phi_1(x) = 1$ with characteristic value $\lambda_1 = 1$. 
    
    Following the procedure outlined in Lecture 8, for $\lambda \not= \lambda_1 = 1$, the unique solution to
    $$
      y = \lambda A y + f
    $$
    is given by
    $$
      y(x) = f(x) +  \lambda \int_0^1 \sum_{k=1}^\infty \frac {\phi_k(s)\phi_k(x)}{\lambda_k-\lambda} f(s)\, ds = f(x) + \frac{\lambda}{1-\lambda} \int_0^1 f(s) ds.
    $$
    Note that this agrees with \eqref{neumann_res} and extends it to $\lambda > 1$.

    If $\lambda = 1$, then there are two cases to consider.  First, if $f$ is orthogonal to $1$, i.e.  $ \int_0^1 f(s)\, ds = 0 $, then $y = f$ since $Af = 0$.  Note that these are not unique. If $f$ is not orthogonal to $1$, then there is no solution.
    \end{solution}

  \subproblem{Use the fact that the kernel is degenerate.}

  \begin{solution}
    Since the kernel is degenerate, we can reduce the problem by setting $c = \int_0^1 y(s)ds$, then integrating both sides of the problem. We get
    $$
      c = \int_0^1 \lambda c + f(s) ds = \lambda c + f_1.
    $$
    where $f_1 = \int_0^1f(s)ds$. Solving this simple linear equation for $c$ when $\lambda \not= 1$, we have $c = (1 - \lambda)^{-1} f_1 $
    so 
    \begin{equation}
      y = \lambda c + f = \frac{\lambda}{1-\lambda} \int_0^1 f(s) ds + f.
    \end{equation}
    Note that this agrees with the previous results.

    When $\lambda =1$, we have the condition that $f_1 = 0$.  So there is no solution when $f_1 \not=0$, and non-unique solutions $y = c + f$ otherwise.  This coincides with the symmetric kernel analysis.
    \vspace{-1em}
  \end{solution}
  \newpage

\problem{Analyze the equation 
$$
  y(x) = \lambda \int_{-1}^1 (2xs^3 + 5x^2s^2)y(s)\, ds + 7x^4 + 3
$$
and solve it for any $\lambda$.
}

\begin{solution}
  Let $A$ denote the integral operator above and $f(x) = 7x^4+3$.  Note that the kernel is degenerate; i.e. it is of the form $\sum_{i=1}^2 a_i(x)b_i(s)$ where $a_1(x) = 2x$, $b_1(s) =  s^3$, $a_2(x)=5x^2$, and $b_2(s)=s^2$. 
  Hence, solutions are of the form
  \begin{equation}
    y(x) = \lambda a_1(x) \Big(y,b_1\Big) + \lambda a_2(x)\Big(y,b_2\Big) + f(x), \label{solution_form}\\
  \end{equation}
  where $\Big( \cdot,\cdot\Big)$ denotes the integral inner product.

  Taking integral inner products on both sides of \eqref{solution_form} by $b_1$ and $b_2$, we have,
  \begin{equation}
  \left.
  \begin{array}{rl}
    \Big(y,b_1\Big) &= \lambda \Big(a_1,b_1\Big)\Big(y,b_1\Big) + \lambda \Big(a_2,b_1\Big)\Big(y,b_2\Big) + \Big(f,b_1\Big)\\
    \Big(y,b_2\Big) &= \lambda \Big(a_1,b_2\Big)\Big(y,b_1\Big) + \lambda \Big(a_2,b_2\Big)\Big(y,b_2\Big) + \Big(f,b_2\Big)
    \end{array}
  \right\} \label{system}
  \end{equation}
  if and only if
  $$
    Y = \lambda KY + F,
  $$
  where $Y,K$ and $F$ are the obvious vectors/matrices of inner products.  The entries of $K$ and $F$ are
  $$
    \renewcommand{\arraystretch}{2}
    K = \begin{bmatrix}
	\Big(a_1,b_1\Big) & \Big(a_2,b_1\Big)\\
	\Big(a_1,b_2\Big) & \Big(a_2,b_2\Big)\\
    \end{bmatrix}
    =
     \begin{bmatrix}
	\int_{-1}^12\,x^4dx  & \int_{-1}^15\,x^5dx\cr 
	\int_{-1}^12\,x^3 dx  & \int_{-1}^15\,x^4 dx\cr 
     \end{bmatrix}
    =
      \begin{bmatrix}
	{\displaystyle \frac{4}{5}}&0\cr 0&2\cr
      \end{bmatrix}, 
  $$
  and
  $$
    \renewcommand{\arraystretch}{2}
    F = 
	\begin{bmatrix}
	 \Big(f,b_1\Big) \\
	 \Big(f,b_2\Big) 
	\end{bmatrix} 
      =
	\begin{bmatrix}
	  \int_{-1}^17\,s^7+3\,s^3ds\cr 
	  \int_{-1}^17\,s^6+3\,s^2ds\cr 
	\end{bmatrix} 
      =
	\begin{bmatrix}
	 0\\
	 4
	\end{bmatrix}.
  $$
  Thus, \eqref{system} reduces to
  \begin{equation}
  \left.
  \begin{array}{rl}
    \Big(y,b_1\Big) &= \frac 45\lambda \Big(y,b_1\Big)\\
    \Big(y,b_2\Big) &= 2\lambda  \Big(y,b_2\Big) + 4
    \end{array}
  \right\}. \label{reduced_system}
  \end{equation}

  When $\lambda = \frac54$, the value of $\Big(y,b_1\Big) = c$ is arbitrary and $\Big(y,b_2\Big) = -\frac 83$, and from \eqref{solution_form} we can write the non-unique solutions as 
  $$
    y(x) = c \frac 52 x  -  \frac {50}3 x^2 + 7x^4 +3.
  $$

  When $\lambda = \frac 12$, then \eqref{reduced_system} gives rise the contradiction $0=4$, and there is no solution.

  When $\lambda\not= \frac 54$ and $\lambda\not= \frac 12$, the coefficients in $Y$ are given by
  $$
    \renewcommand{\arraystretch}{2.5}
    Y = (I-\lambda K)^{-1} F = 
    \begin{bmatrix}
      {\displaystyle \frac{1}{1-  \frac 45\lambda}} & 0\\
      0 & {\displaystyle \frac{1}{1-  2\lambda}}\\
    \end{bmatrix}
    \begin{bmatrix}
     0\\
     4
    \end{bmatrix}
      = 
    \begin{bmatrix}
      0\\
      {\displaystyle \frac{4}{1-  2\lambda}}\\
    \end{bmatrix}
  $$
  and finally substituting into \eqref{solution_form} gives
  \begin{align*}
    y(x) = \frac{4\lambda }{1-  2\lambda} 5x^2 + 7x^4 + 3.
  \end{align*}
\end{solution}
\end{document} 


